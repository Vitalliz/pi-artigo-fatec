\documentclass[
    landscape,
    a4paper,
    12pt,
    english,
    brazilian,
]{article}

\usepackage{fatec-article}
\usepackage{array} % Necessário para 'm{width}' para centralização vertical
\usepackage{longtable} % Necessário para tabelas que ocupam várias páginas
\usepackage{setspace}

\begin{document}

\section*{Diário de Bordo}

% Usando o ambiente longtable para tabelas que se estendem por várias páginas
\begin{longtable}{|m{4cm}|m{2.8cm}|m{2.8cm}|m{4.8cm}|m{8cm}|} % Usando 'm' para centralizar verticalmente
    \hline
    Nome da Atividade & Data de início & Data de término & Responsável pela atividade & Descrição da atividade realizada \\ \hline
    \endfirsthead
    % Cabeçalho que aparecerá no início de cada página
    \hline
    Nome da Atividade & Data de início & Data de término & Responsável pela atividade & Descrição da atividade realizada \\ \hline
    \endhead
    
    % Aqui você adiciona as linhas da tabela
    \centering Revisão do Artigo & \centering 08/10/2024 & \centering 17/11/2024 & \centering Luiz & Foram corrigidos erros de português, revisados os objetivos e reformulada a seção de estado da arte. Além disso, incluíram-se os resultados preliminares com as telas do site, o modelo físico do banco de dados e explicações sobre o diagrama de classes e objetos na seção de resultados. Por fim, algoritmos de recursividade foram implementados na tela de busca do site.\\ \hline
    \centering Prototipação telas do site no Figma& \centering 05/10/2024 & \centering 17/11/2024 & \centering Amanda & Desenvolvimento das telas iniciais com base na prototipação do semestre anterior, das telas da versão mobile, com adição de landing page, reformulando o design anterior para uma versão desktop web. Criação de componentes afim de diminuir a quantidade de telas feitas, e deixar mais eficiente, melhora no design para facilitar a visualização da simulação.\\ \hline
    \centering Desenvolvimeto da parte front-end do site & \centering 15/10/2024 & \centering 17/11/2024 & \centering Amanda e Valéria & Criação das telas web, ficando fiel à as telas prototipadas no figma, o desenvolvimento do design foi feito utilizando CSS e Bootstrap\\ \hline
    \centering Desenvolvimento da parte back-end do site & \centering 15/10/2024 & \centering 17/11/2024 & \centering Todos do grupo participaram dessa parte & Foram feitas as funcionalidades do Prototipo Figma, sistemas internos de rota, integração com o banco de dados, utilizando as bibliotecas e funcionalidades do Node.js, sendo elas Express, Nodemon, Middleware, View Engine EJS, Sequelize(utilizado na integração do banco de dados), mysql2, utilizando a arquitetura MVC para deixar as pastas do código estruturadas. \\ \hline
    \centering Banco de dados Físico& \centering 25/10/2024 & \centering 10/11/2024 & \centering Lucas & Foi feito o banco de dados que foi utilizado no site, o banco de dados está de acordo com todas as funcionalidades que o projeto terá.\\ \hline
    \centering DER do Banco de dados& \centering 23/10/2024 & \centering 25/10/2024 & \centering Lucas & O DER do banco de dados foi refeito, agora com todos os dados corretos e alinhados ao projeto atual.\\ \hline
    \centering Diagrama de Classe & \centering 25/09/2024 & \centering 17/11/2024 & \centering Luiz & Foi feito o diagrama de classe do projeto, baseado nas funções que o projeto terá.\\ \hline
    \centering Diagrama de Objetos & \centering 25/09/2024 & \centering 17/11/2024 & \centering Luiz & Foi feito o diagrama de objetos do projeto, mostrando os objetos das classes importantes.\\ \hline
    \centering Diagrama Caso de Uso & \centering 11/11/2024 & \centering 17/11/2024 & \centering Valéria & Foi refeito o diagrama de caso de uso, com todos os atores e suas respectivas ações corretas e alinhadas com o projeto atual.\\ \hline
    \centering Banner & \centering 13/11/2024 & \centering 18/11/2024 & \centering Amanda & Feito o banner que será utilizado para a próxima feira tecnólogica.\\ \hline
    % Continue adicionando linhas conforme necessário
    
\end{longtable}

\end{document}
