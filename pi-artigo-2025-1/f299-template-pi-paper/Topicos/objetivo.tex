A agricultura de citros enfrenta diversos tipos de doenças e deficiências nutricionais que comprometem o desenvolvimento saudável das plantas. Essas condições podem ser identificadas visualmente por alterações nas folhas, cascas e frutos, porém o diagnóstico tradicional por meio de análises químicas é demorado, exige coleta física e depende de infraestrutura especializada. A utilização de inteligência artificial (IA) para detectar padrões visuais oferece uma alternativa acessível e eficaz, permitindo diagnósticos em tempo real diretamente no campo.

O projeto tem como objetivo principal desenvolver um sistema inteligente baseado em visão computacional para identificar deficiências de manganês e cobre em folhas de Citrus reticulata (mexerica), por meio da análise de imagens capturadas com dispositivos móveis. O sistema visa proporcionar agilidade, precisão e autonomia ao agricultor, contribuindo para a sustentabilidade da produção e a redução de perdas.

\textbf{Os objetivos principais do projeto são:}
\begin{enumerate}
\item Aprofundar na coleta e organização de informações sobre os sintomas visuais das deficiências nutricionais específicas da mexerica, com foco em cobre e manganês.
\item Desenvolver um banco de dados com imagens rotuladas de folhas apresentando diferentes níveis de deficiência, com expectativa de compor uma base robusta e representativa.
\item Realizar o treinamento da IA utilizando uma arquitetura de rede neural convolucional (CNN), aplicando técnicas de aumento de dados (data augmentation), validação cruzada e ajuste de hiperparâmetros.
\item Implementar uma solução funcional que possibilite ao agricultor, ao apontar a câmera do celular para a folha, receber um diagnóstico automatizado com base em IA e visão computacional.
\item Avaliar a precisão do sistema desenvolvido, comparando-o com métodos tradicionais de análise foliar e de solo, para verificar sua eficácia no ambiente agrícola real.
\end{enumerate}

\textbf{Além disso, o projeto também contempla os seguintes objetivos específicos:}
\begin{enumerate}
\item Habilitar o registro dos diagnósticos no sistema, permitindo ao usuário cadastrar o número da planta e o talhão, facilitando o acompanhamento e controle nutricional.
\item Implementar uma funcionalidade de histórico, permitindo ao agricultor visualizar a evolução do estado das plantas e compará-lo com registros anteriores.
\item Criar um módulo de recomendações técnicas, com sugestões agronômicas baseadas nos resultados obtidos, visando a aplicação mais assertiva de insumos.
\item Prever, para fases futuras, o uso de drones para captura de imagens aéreas, com o objetivo de identificar áreas críticas da plantação.
\item Incluir, posteriormente, funcionalidades como mapas interativos e mapas de calor, que indiquem visualmente a distribuição das deficiências por talhão, facilitando a tomada de decisão.
\item Explorar a possibilidade de expansão da plataforma para outras culturas e deficiências nutricionais, visando aumentar o escopo e escalabilidade da solução.
\item Contribuir para a sustentabilidade agrícola por meio da redução de perdas por deficiência nutricional e aumento da rentabilidade dos produtores rurais por meio do uso direcionado de insumos.
\end{enumerate}