A escassez de alimentos ainda é um desafio global, agravado por fatores como pobreza, conflitos e mudanças climáticas. Como resposta, a Organização das Nações Unidas (ONU) estabeleceu, em 2015, os Objetivos de Desenvolvimento Sustentável (ODS), entre eles o ODS 2 — Fome Zero e Agricultura Sustentável, que busca garantir segurança alimentar por meio do aumento da produtividade agrícola e do uso de tecnologias sustentáveis (Fundecitrus, 2018).

No Brasil, um dos entraves à produtividade agrícola é a deficiência nutricional nas plantas, especialmente em culturas como a Citrus reticulata (mexerica), bastante suscetível a desequilíbrios minerais. Tais deficiências podem afetar diretamente o crescimento e o rendimento da produção, prejudicando pequenos e médios agricultores (Bueno; Gasparotto, 1999).

Entre os problemas fitossanitários que atingem os citros, destaca-se o greening (HLB), uma doença sem cura que compromete severamente a produção. Embora o foco deste projeto não seja essa doença, é importante mencionar que seus sintomas visuais, como clorose e deformações nas folhas, podem ser confundidos com deficiências nutricionais, como as de manganês e cobre — foco deste estudo. Isso torna o diagnóstico preciso ainda mais relevante (Fundecitrus, 2018; Aregbe; Farnsworth; Simnitt, 2024).

A deficiência de manganês causa clorose internerval nas folhas jovens, enquanto a de cobre pode levar ao encurtamento dos ramos, folhas pequenas e gomose nos frutos. Ambas influenciam diretamente a produtividade e estão associadas ao pH do solo — sendo a carência de manganês mais comum em solos ácidos e a de cobre em solos mais alcalinos (Bruna, 2019; Machado, 2022).

Considerando esse cenário, este projeto propõe o desenvolvimento de um sistema baseado em Inteligência Artificial (IA) para auxiliar produtores rurais na identificação rápida e precisa dessas deficiências em folhas de mexerica. Por meio da análise de imagens capturadas via smartphones, o sistema utilizará visão computacional, com suporte de redes neurais convolucionais (CNNs) e técnicas de aprendizado profundo (Deep Learning), para classificar a saúde foliar de forma prática, acessível e em tempo real (Qin et al., 2018).

Ao integrar essas tecnologias ao contexto da agricultura nacional, o projeto visa contribuir diretamente com os objetivos do ODS 2, promovendo práticas agrícolas mais sustentáveis, reduzindo perdas na produção e fortalecendo a segurança alimentar por meio da inovação tecnológica (Christin; Hervet; Lecomte, 2019).