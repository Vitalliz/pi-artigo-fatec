O desenvolvimento do projeto será conduzido em etapas que seguem a metodologia de desenvolvimento ágil Scrum, permitindo uma adaptação flexível aos requisitos ao longo do processo. Além disso, será utilizada a linguagem UML (Unified Modeling Language) para a elaboração de diagramas de classes, objetos e casos de uso, bem como a aplicação da análise SWOT (Forças, Fraquezas, Oportunidades e Ameaças), a fim de garantir uma compreensão completa da estrutura e do comportamento do sistema.

As principais etapas do projeto são descritas a seguir:

\begin{itemize}
    \item \textbf{1.1 Coleta de Requisitos e Design}: A ferramenta Figma será utilizada para o desenvolvimento de protótipos de baixa e alta fidelidade das interfaces da aplicação. Essa escolha se justifica pela facilidade de uso da plataforma, sua capacidade colaborativa em tempo real e pelos recursos que permitem validar aspectos funcionais e de usabilidade com clareza e agilidade.
    \item \textbf{1.2 Desenvolvimento Front-end}: O código da interface será estruturado com HTML, CSS e React, utilizando o framework Next.js, que oferece recursos avançados como renderização do lado do servidor (SSR), geração de páginas estáticas (SSG) e roteamento dinâmico. Essa abordagem permite maior performance, melhor usabilidade e organização otimizada do projeto. O JavaScript será utilizado para implementar a lógica de interação entre o usuário e a interface, garantindo dinamismo e fluidez. O Node.js será adotado em conjunto para a criação de componentes reutilizáveis e integração com o back-end, aumentando a eficiência e a modularidade do código.
    \item \textbf{1.3 Desenvolvimento Back-end e Banco de Dados}: O DER (Diagrama Entidade-Relacionamento) do banco de dados será modelado utilizando o brModelo, e a implementação será feita com as ferramentas MySQL WorkBench e XAMPP, garantindo uma estrutura robusta e confiável para o armazenamento dos dados. A integração entre o front-end e o back-end será realizada utilizando Node.js, responsável por criar e gerenciar as rotas da aplicação, permitindo a comunicação entre os módulos visuais e os serviços do sistema. Para os componentes que envolvem visão computacional, será utilizada a linguagem Python, aplicada na análise de imagens por meio de algoritmos de inteligência artificial. No ambiente mobile, será utilizado React Native, devido à sua natureza multiplataforma e à possibilidade de criação de APIs eficientes com bom desempenho em dispositivos móveis.
    \item \textbf{1.4 Treinamento e Avaliação de Algoritmos de IA}: O modelo de IA será baseado em redes neurais convolucionais (CNNs), treinado com imagens rotuladas de folhas de mexerica contendo sintomas de deficiência nutricional. Serão aplicadas técnicas de data augmentation, como variação de brilho, ângulo e contraste, para ampliar a diversidade do conjunto de dados. A eficiência do algoritmo será mensurada por meio de análise de desempenho, utilizando métricas como acurácia, precisão, sensibilidade, F1-score e tempo de execução. A abordagem adotada seguirá o modelo de avaliação empregado por Tran et al. (2019), que compararam arquiteturas como Inception-ResNet v2, Autoencoder CNN e técnicas de Ensemble Averaging para detecção de deficiências nutricionais em folhas de tomate. Da mesma forma, pretende-se validar o modelo proposto com técnicas de validação cruzada e métricas de desempenho estatístico, de forma a garantir confiabilidade e precisão no diagnóstico das deficiências de cobre e manganês.
    \item \textbf{1.5 Testes e Validação}: Após a implementação do sistema, serão conduzidos testes automatizados e manuais para verificar o correto funcionamento de cada componente. Serão utilizadas ferramentas como o Selenium, que permite a automação de testes de interface, com foco na experiência do usuário e na funcionalidade dos elementos visuais da aplicação. Os testes manuais complementarão essa abordagem, possibilitando a verificação de cenários não cobertos pelos testes automatizados.
    \item \textbf{1.6 Publicação e Acompanhamento}: O sistema será inicialmente publicado na plataforma web, com foco no desenvolvimento do front-end e back-end. Em uma etapa futura, funcionalidades presentes na versão web serão adaptadas para o ambiente mobile, permitindo que os agricultores utilizem o aplicativo diretamente no campo para escanear folhas em tempo real e receber diagnósticos imediatos. A arquitetura do sistema será planejada com foco em escalabilidade e manutenção contínua, com atualizações baseadas no feedback dos usuários. Essa abordagem progressiva permitirá a constante evolução do sistema, de forma a atender às reais demandas do setor agrícola e garantir a longevidade e eficácia da solução proposta.
\end{itemize}

O fluxograma do processo desenvolvido pode ser consultado no Apêndice B.
