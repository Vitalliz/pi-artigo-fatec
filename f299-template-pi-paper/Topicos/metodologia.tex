\section{Metodologia}

O desenvolvimento do projeto será conduzido em etapas que seguem a metodologia de desenvolvimento ágil (\textit{Scrum}), permitindo uma adaptação flexível aos requisitos ao longo do processo. Além disso, a Linguagem UML será utilizada para elaborar os diagramas de classes, objetos e casos de uso. As etapas principais incluem:

\subsection{Coleta de requisitos e design}
O \textit{Figma} será utilizado para o desenvolvimento de protótipos de baixa e alta fidelidade das interfaces da aplicação, garantindo uma visualização clara dos requisitos funcionais e de usabilidade.

\subsection{Desenvolvimento front-end}
O código será estruturado com \textit{HTML} e \textit{CSS} para a definição de layout e estilo, enquanto o \textit{JavaScript} será utilizado para proporcionar interatividade. O \textit{Node.js} será empregado para a criação de componentes reutilizáveis, melhorando a eficiência e a modularidade do código.

\subsection{Desenvolvimento back-end e banco de dados}
O DER do banco de dados será modelado no \textit{brModelo} e implementado no \textit{MySQL WorkBench} e \textit{XAMPP}, garantindo a correta estruturação dos dados. A interação entre \textit{front-end} e \textit{back-end} será implementada utilizando \textit{Python} para a integração da visão computacional e \textit{Node.js} no computador. Para o \textit{mobile}, o \textit{React Native} será utilizado para criar uma API eficiente.

\subsection{Testes e validação}
Após a implementação, serão realizados testes automatizados e manuais para validar o funcionamento correto de cada parte do sistema. Ferramentas como \textit{Selenium} poderão ser utilizadas para automatizar os testes de interface.

\subsection{Publicação e acompanhamento}
O sistema será inicialmente implementado na plataforma \textit{web}, com desenvolvimento do \textit{front-end} e \textit{back-end}. Em uma versão futura, algumas funções disponíveis na versão \textit{web} serão adaptadas para o \textit{mobile}, permitindo o uso do aplicativo no campo para escanear folhas em tempo real e receber feedback instantâneo. Serão garantidas fácil escalabilidade e manutenção, com ajustes realizados com base no feedback dos usuários após a implantação.
