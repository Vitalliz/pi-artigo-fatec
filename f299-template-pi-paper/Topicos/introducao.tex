A escassez de alimentos continua sendo um desafio global, agravado por fatores como pobreza, conflitos e mudanças climáticas. Em resposta, a Organização das Nações Unidas (ONU) instituiu, em 2015, os Objetivos de Desenvolvimento Sustentável (ODS). Entre eles, o ODS 2 — Fome Zero e Agricultura Sustentável — busca assegurar a segurança alimentar por meio do aumento da produtividade agrícola e do uso de tecnologias sustentáveis (Fundecitrus, 2018).

No Brasil, a produtividade agrícola ainda enfrenta barreiras significativas, entre as quais se destacam as deficiências nutricionais das plantas, especialmente em culturas como a \textit{Citrus reticulata} — popularmente conhecida como mexerica. Essa espécie apresenta elevada suscetibilidade a desequilíbrios minerais, o que compromete seu desenvolvimento e reduz o rendimento da produção, afetando, sobretudo, pequenos e médios produtores (Bueno; Gasparotto, 1999).

Durante a pesquisa de campo realizada em propriedades rurais produtoras de mexerica, foi observada alta incidência de deficiência de manganês e baixa de cobre, além de numerosos casos de greening (HLB) distribuídos por toda a plantação. Embora o foco deste estudo não seja essa doença, sua menção é relevante, pois os sintomas visuais — como clorose e deformações foliares — podem ser confundidos com deficiências nutricionais, o que reforça a importância de diagnósticos precisos e acessíveis (Fundecitrus, 2018; Aregbe; Farnsworth; Simnitt, 2024).

A carência de manganês manifesta-se por clorose internerval em folhas jovens, enquanto a deficiência de cobre provoca encurtamento dos ramos, folhas pequenas e ocorrência de gomose nos frutos. Essas condições afetam diretamente a produtividade e estão associadas ao pH do solo — sendo o manganês mais escasso em solos ácidos e o cobre em solos mais alcalinos (Bruna, 2019; Machado, 2022).

Diante desse contexto, este projeto propõe o desenvolvimento de um sistema baseado em Inteligência Artificial (IA) para auxiliar produtores rurais na identificação rápida e precisa dessas deficiências em folhas de mexerica. A solução utilizará imagens capturadas por smartphones e aplicará técnicas de visão computacional, com o suporte de redes neurais convolucionais (CNNs) e \textit{deep learning}, a fim de classificar o estado nutricional das folhas de forma prática, acessível e em tempo real (Qin et al., 2018).

Ao integrar tais tecnologias ao contexto agrícola brasileiro, o projeto busca contribuir com os objetivos do ODS 2 — Fome Zero e Agricultura Sustentável — promovendo práticas mais sustentáveis, reduzindo perdas na produção e fortalecendo a segurança alimentar por meio da inovação tecnológica (Christin; Hervet; Lecomte, 2019).