%%%% fatec-article.tex, 2024/03/10

%% Classe de documento
\documentclass[
  a4paper,%% Tamanho de papel: a4paper, letterpaper (^), etc.
  12pt,%% Tamanho de fonte: 10pt (^), 11pt, 12pt, etc.
  english,%% Idioma secundário (penúltimo) (>)
  brazilian,%% Idioma primário (último) (>)
]{article}

%% Pacotes utilizados
\usepackage[]{fatec-article}
\usepackage{float}

\Author{1}{Name={ Fagundes. L \\ Freitas. A \\ Freitas. V\\ Medina. L}}

\Author{2}{Name={\{ lucas.fagundes3@fatec.sp.gov.br \}\\ \{ amanda.freitas14@fatec.sp.gov.br \} \\ \{ valeria.freitas@fatec.sp.gov.br \} \\ \{ luiz.medina@fatec.sp.gov.br \}}}

%% Definição das palavras-chaves/keywords
\Keyword{1}{Citrus reticulata}{Citrus reticulata}
\Keyword{2}{Deficiência nutricional}{Nutritional deficiency}
\Keyword{3}{Visão computacional}{Computer Vision}
\Keyword{4}{Manganês}{Manganese}
\Keyword{5}{Cobre}{Copper}

%%%% Resumo no idioma primário (brazilian)
\begin{Abstract}[brazilian]%% Idioma (brazilian ou english)
  Este trabalho apresenta o desenvolvimento inicial do projeto NitrusLeaf, que utiliza técnicas de Inteligência Artificial (IA) para identificar deficiências nutricionais de cobre e manganês nas folhas da citrus reticulata (mexerica), alinhando-se com o Objetivo de Desenvolvimento Sustentável (ODS) 2 - Fome zero e agricultura sustentável da Agenda 2030 da ONU. O objetivo é reduzir as perdas na produção agrícola e promover a sustentabilidade. A aplicação visa facilitar aos agricultores a identificação desses problemas de forma rápida e precisa, utilizando visão computacional para analisar imagens das folhas, capturadas por drones ou enviadas pelos agricultores. A IA será treinada com o uso de redes neurais convolucionais (CNNs) utilizando um banco de dados de imagens de folhas com e sem deficiências. Espera-se que o NitrusLeaf se torne uma ferramenta eficaz para a agricultura sustentável após a conclusão desta pesquisa.  
\end{Abstract}

%%%% Resumo no idioma secundário (english)
\begin{Abstract}[english]%% Idioma (brazilian ou english)
  This work presents the initial development of the NitrusLeaf project, which utilizes specific Artificial Intelligence (AI) techniques to identify copper and manganese nutritional deficiencies in citrus reticulata (tangerine) leaves, with the aim of following the UN's 2030 Agenda, specifically Sustainable Development Goal (SDG) 2 Zero Hunger and Sustainable Agriculture, seeking to reduce agricultural production losses and promote sustainability. This application aims to help farmers identify these problems more easily and quickly, reducing tangerine crop losses and increasing plant health. Through computer vision, the system analyzes images of leaves, taken by drones or sent by farmers, and provides an accurate diagnosis. Convolutional Neural Networks (CNNs) will be used to train the AI with a database of images of leaves with and without deficiencies. After further research, it is expected that NitrusLeaf will become an effective tool for sustainable agriculture.
\end{Abstract}

%% Processamento de entradas (itens) do índice remissivo (makeindex)
\makeindex%

%% Arquivo(s) de referências
\addbibresource{fatec-article.bib}

%% Início do documento
\begin{document}

% Seções e subseções
%\section{Título de Seção Primária}%

%\subsection{Título de Seção Secundária}%

%\subsubsection{Título de Seção Terciária}%

%\paragraph{Título de seção quaternária}%

%\subparagraph{Título de seção quinária}%

\section*{Introdução}%
\label{sect:intro}
A escassez de alimentos continua sendo um desafio global, agravado por fatores como pobreza, conflitos e mudanças climáticas. Em resposta, a Organização das Nações Unidas (ONU) instituiu, em 2015, os Objetivos de Desenvolvimento Sustentável (ODS). Entre eles, o ODS 2 — Fome Zero e Agricultura Sustentável — busca assegurar a segurança alimentar por meio do aumento da produtividade agrícola e do uso de tecnologias sustentáveis.

No Brasil, um dos principais entraves à produtividade agrícola está relacionado às deficiências nutricionais nas plantas, especialmente em culturas cítricas como a \textit{Citrus reticulata} (mexerica), que apresentam elevada sensibilidade a desequilíbrios minerais. Essas deficiências afetam diretamente o desenvolvimento vegetativo e o rendimento da produção, refletindo-se em perdas econômicas para pequenos e médios produtores. Estudos recentes indicam que a concentração de nutrientes em plantas cítricas varia sazonalmente no solo, nas folhas e na seiva do xilema, o que dificulta o diagnóstico preciso do estado nutricional e contribui para manejos inadequados de adubação e correção do solo \cite{FaveroFilho2022}. 

Durante a pesquisa de campo realizada em propriedades rurais produtoras de mexerica, foi observada alta incidência de deficiência de manganês e baixa de cobre, além de numerosos casos de greening (HLB) distribuídos por toda a plantação. Embora o foco deste estudo não seja essa doença, sua menção é relevante, pois os sintomas visuais — como clorose e deformações foliares — podem ser confundidos com deficiências nutricionais, o que reforça a importância de diagnósticos precisos e acessíveis \cite{Fundecitrus2021, Aregbe2024}.

A carência de manganês manifesta-se por clorose internerval em folhas jovens, enquanto a deficiência de cobre provoca encurtamento dos ramos, folhas pequenas e ocorrência de gomose nos frutos. Essas condições afetam diretamente a produtividade e estão associadas ao pH do solo — sendo o manganês mais escasso em solos ácidos e o cobre em solos mais alcalinos \cite{Bruna2019, Machado2022}.

Durante a última década, técnicas de Inteligência Artificial (IA) e \textit{deep learning} têm sido cada vez mais aplicadas na agricultura de precisão, principalmente no diagnóstico de deficiências nutricionais em plantas. Entre os métodos de aprendizado de máquina, as redes neurais convolucionais (CNNs), uma das principais arquiteturas de \textit{deep learning}, destacam-se por sua capacidade de extrair automaticamente características visuais complexas de imagens, sem a necessidade de pré-processamento extensivo ou seleção manual de atributos. Diferentemente de algoritmos clássicos, como SVMs, árvores de decisão ou k‑NN, que dependem fortemente de variáveis pré-definidas e da engenharia de características, as CNNs conseguem aprender representações hierárquicas diretamente a partir das imagens das folhas, capturando padrões sutis de clorose, deformações ou manchas associadas a deficiências nutricionais \cite{Qin2018, Christin2019}.

Ao integrar tais tecnologias ao contexto agrícola brasileiro, o projeto busca contribuir com os objetivos do ODS 2 — Fome Zero e Agricultura Sustentável — promovendo práticas mais sustentáveis, reduzindo perdas na produção e fortalecendo a segurança alimentar por meio da inovação tecnológica.

\section*{OBJETIVO} \label{sect:obj}

A agricultura de citros enfrenta diversos tipos de doenças e deficiências nutricionais que comprometem o desenvolvimento saudável das plantas. Essas condições podem ser identificadas visualmente por alterações nas folhas, cascas e frutos, porém o diagnóstico tradicional por meio de análises químicas é demorado, exige coleta física e depende de infraestrutura especializada. A utilização de inteligência artificial (IA) para detectar padrões visuais oferece uma alternativa acessível e eficaz, permitindo diagnósticos em tempo real diretamente no campo.

O projeto tem como objetivo principal desenvolver um sistema inteligente baseado em visão computacional para identificar deficiências de manganês e cobre em folhas de Citrus reticulata (mexerica), por meio da análise de imagens capturadas com dispositivos móveis. O sistema visa proporcionar agilidade, precisão e autonomia ao agricultor, contribuindo para a sustentabilidade da produção e a redução de perdas.

\textbf{Os objetivos principais do projeto são:}
\begin{enumerate}
\item Aprofundar na coleta e organização de informações sobre os sintomas visuais das deficiências nutricionais específicas da mexerica, com foco em cobre e manganês.
\item Desenvolver um banco de dados com imagens rotuladas de folhas apresentando diferentes níveis de deficiência, com expectativa de compor uma base robusta e representativa.
\item Realizar o treinamento da IA utilizando uma arquitetura de rede neural convolucional (CNN), aplicando técnicas de aumento de dados (data augmentation), validação cruzada e ajuste de hiperparâmetros.
\item Implementar uma solução funcional que possibilite ao agricultor, ao apontar a câmera do celular para a folha, receber um diagnóstico automatizado com base em IA e visão computacional.
\item Avaliar a precisão do sistema desenvolvido, comparando-o com métodos tradicionais de análise foliar e de solo, para verificar sua eficácia no ambiente agrícola real.
\end{enumerate}

\textbf{Além disso, o projeto também contempla os seguintes objetivos específicos:}
\begin{enumerate}
\item Habilitar o registro dos diagnósticos no sistema, permitindo ao usuário cadastrar o número da planta e o talhão, facilitando o acompanhamento e controle nutricional.
\item Implementar uma funcionalidade de histórico, permitindo ao agricultor visualizar a evolução do estado das plantas e compará-lo com registros anteriores.
\item Criar um módulo de recomendações técnicas, com sugestões agronômicas baseadas nos resultados obtidos, visando a aplicação mais assertiva de insumos.
\item Prever, para fases futuras, o uso de drones para captura de imagens aéreas, com o objetivo de identificar áreas críticas da plantação.
\item Incluir, posteriormente, funcionalidades como mapas interativos e mapas de calor, que indiquem visualmente a distribuição das deficiências por talhão, facilitando a tomada de decisão.
\item Explorar a possibilidade de expansão da plataforma para outras culturas e deficiências nutricionais, visando aumentar o escopo e escalabilidade da solução.
\item Contribuir para a sustentabilidade agrícola por meio da redução de perdas por deficiência nutricional e aumento da rentabilidade dos produtores rurais por meio do uso direcionado de insumos.
\end{enumerate}

\section*{ESTADO DA ARTE} \label{sect:estadoarte}

Entre os muitos desafios enfrentados pelos agricultores, a deficiência de minerais nas plantas é uma preocupação significativa, pois pode resultar em perdas de produtividade e qualidade dos cultivos. A mexerica (Citrus reticulata) é uma das culturas suscetíveis a deficiências minerais, o que pode afetar seu crescimento, desenvolvimento e produção.

O artigo de \textcite{EstadoArte1} discute avanços recentes nas tecnologias de visão computacional, aprendizado de máquina (ML) e aprendizado profundo (DL), que têm sido aplicadas ao monitoramento agrícola para melhorar a produtividade e a qualidade das colheitas. Entre as tecnologias abordadas, destacam-se imagens de satélite, sensoriamento remoto, Internet das Coisas (IoT), dispositivos de sensor e Veículos Aéreos Não Tripulados (VANTs). Esses sistemas são utilizados para capturar dados visuais e detectar deficiências nutricionais em tempo real, permitindo um diagnóstico precoce e aumentando a eficiência na aplicação de insumos agrícolas.

O artigo cita trabalhos que demonstram como a visão computacional, combinada com ML e DL, pode identificar padrões visuais como coloração, textura e bordas em imagens de plantas. Esse exemplo ilustra como essas técnicas permitem um diagnóstico não invasivo, usando câmeras digitais e algoritmos avançados para diferenciar entre folhas saudáveis e folhas com deficiência nutricional.

Com base nesses avanços, é viável que o projeto NitrusLeaf implemente um sistema semelhante, utilizando smartphones e visão computacional para capturar e analisar imagens de folhas de mexerica. Com o suporte de algoritmos de ML/DL, o sistema pode processar essas imagens para identificar deficiências específicas, como de cobre e manganês, diretamente no campo.

O artigo de \cite{EstadoArte2} explora um pipeline detalhado \Cref{fig:pipeline} para identificar doenças e deficiências nutricionais em plantas, baseado em técnicas de processamento de imagem. Esse pipeline, ou sequência organizada de etapas, permite a análise automatizada e eficiente de imagens de plantas, produzindo diagnósticos agrícolas precisos. O pipeline descrito no artigo inclui as seguintes fases: Aquisição de imagem, Pré-processamento, Segmentação, Extração de características, Classificação e, finalmente, Detecção e Diagnóstico. Cada fase depende da anterior e contribui para refinar e analisar os dados, de modo a produzir uma resposta confiável sobre a condição da planta.

\begin{itemize} 
    \item \textbf{Aquisição de imagem}: Envolve a captura de imagens de plantas utilizando câmeras, drones (Veículos Aéreos Não Tripulados - VANTs) ou dispositivos móveis. Essa etapa assegura que as imagens tenham qualidade suficiente para as fases subsequentes do processamento.
    \item \textbf{Pré-processamento}: Aqui, técnicas de correção de ruído, ajuste de contraste e brilho, além de redimensionamento e rotação, são aplicadas para melhorar a qualidade da imagem e facilitar a segmentação. 
    \item \textbf{Segmentação de imagem}: É feita para isolar a folha ou parte relevante da planta, separando-a do fundo. Técnicas comuns incluem limiarização, segmentação por cor e abordagens de aprendizado de máquina. \item \textbf{Extração de características}: Essa etapa analisa e extrai informações importantes, como cor, textura e forma, que são essenciais para diferenciar entre folhas saudáveis e afetadas. Métodos como histogramas de cor e a Matriz de Co-ocorrência de Níveis de Cinza (GLCM) são comumente usados. 
    \item \textbf{Classificação}: Modelos de aprendizado de máquina ou redes neurais profundas (como CNNs e SVMs) são empregados para classificar as imagens, diferenciando as folhas saudáveis das que apresentam deficiências ou doenças. 
    \item \textbf{Detecção e Diagnóstico}: A última etapa envolve o diagnóstico da condição da planta, onde é identificada a doença ou deficiência nutricional específica, e recomendações são geradas com base nos resultados da classificação.
\end{itemize}

\begin{figure}[H]
    \centering
    \caption{Esquema do pipeline para detecção de doenças em plantas baseada em imagens}
    \includegraphics[width=0.8\linewidth]{Illustrations/pipeline.png}
    \label{fig:pipeline}
    \SourceOrNote{Adaptado de \cite{EstadoArte2}}
    \end{figure}


    Além disso, o artigo explora técnicas não intrusivas para a detecção de deficiências nutricionais, utilizando métodos avançados de análise de imagens. Por exemplo, uma tabela destaca o uso do sistema de cores RGB (Red, Green, Blue) para identificar deficiências de nutrientes, incluindo a deficiência de manganês em citrus. Essa abordagem reforça a viabilidade do projeto NitrusLeaf, que tem como objetivo identificar deficiências de cobre e manganês nas folhas de mexerica. Ao adotar essas tecnologias e adaptar o pipeline descrito, o NitrusLeaf poderá se consolidar como uma solução eficaz e prática para diagnósticos agrícolas baseados em visão computacional.



O artigo de \textcite{EstadoArte3} utiliza os modelos Inception-ResNet v2, Autoencoder de Rede Neural Convolucional (CNN), e uma combinação desses dois modelos por meio de Ensemble Averaging para melhorar a detecção precoce de deficiências nutricionais de cálcio, nitrogênio e potássio em plantações de tomate. A identificação rápida dessas deficiências é essencial, pois a falta de intervenção pode levar a condições mais graves, incluindo doenças que afetam a produtividade e a saúde das plantas. Para garantir que os modelos fossem treinados com imagens robustas e representativas, os autores aplicaram técnicas de pré-processamento e aumento de dados (data augmentation), como ajuste de ângulo, brilho e contraste. Essas técnicas aumentaram a diversidade visual do conjunto de dados, ajudando os modelos a generalizar melhor e a detectar padrões de deficiência com mais precisão.

A escolha dos modelos Inception-ResNet v2 e Autoencoder foi motivada pela capacidade dessas arquiteturas em capturar características visuais complexas nas folhas e nos frutos, essenciais para distinguir entre deficiências nutricionais que compartilham sintomas visuais semelhantes. O Inception-ResNet v2, por exemplo, combina as vantagens das redes Inception e ResNet, permitindo uma análise detalhada de padrões locais e globais na imagem. Já o Autoencoder oferece uma estrutura de compressão e reconstrução útil para destacar anomalias visuais, como as causadas por deficiências. O estudo usou um conjunto de 571 imagens de tomates cultivados em estufas, das quais 80\% (461 imagens) foram destinadas ao treinamento e 20\% (110 imagens) para a validação dos modelos. Após extensivos testes, os resultados mostraram que o Inception-ResNet v2 obteve uma precisão de 87,27\%, o Autoencoder alcançou 79,09\%, e a técnica de Ensemble Averaging conseguiu uma precisão de 91\%.

As abordagens e metodologias deste artigo são relevantes para nosso projeto, pois ambos compartilham o foco em visão computacional e aprendizado profundo para a detecção de deficiências nutricionais em plantas. Enquanto o artigo de Tran et al. se concentra em deficiências de nitrogênio e potássio em folhas de tomate, nosso projeto utiliza técnicas similares para identificar deficiências de cobre e manganês em folhas de mexerica. Assim, este trabalho oferece uma base metodológica útil, especialmente na escolha de modelos de CNN e na importância de uma abordagem de ensemble para melhorar a precisão no diagnóstico.

Esses resultados indicam que, com o uso apropriado das tecnologias discutidas, nosso projeto NitrusLeaf poderá alcançar resultados satisfatórios. Aproveitando as tecnologias descritas, buscamos oferecer uma funcionalidade avançada ao sistema, permitindo a identificação precisa de deficiências minerais, como as de manganês e cobre, por meio da análise de imagens capturadas pelos usuários.

\section*{METODOLOGIA} \label{sect:metodologia}

\section{Metodologia}

O desenvolvimento do projeto será conduzido em etapas que seguem a metodologia de desenvolvimento ágil (Scrum), permitindo uma adaptação flexível aos requisitos ao longo do processo. As etapas principais incluem:

\subsection{Coleta de requisitos e design}
Usaremos o Figma para desenvolver protótipos de baixa e alta fidelidade das interfaces da aplicação, garantindo uma visualização clara dos requisitos funcionais e de usabilidade.

\subsection{Desenvolvimento front-end}
O código será estruturado com HTML e CSS para a definição de layout e estilo, enquanto o JavaScript será utilizado para proporcionar interatividade. O \textit{Node.js} será empregado para a criação de componentes reutilizáveis, melhorando a eficiência e a modularidade do código.

\subsection{Desenvolvimento back-end e banco de dados}
O banco de dados será modelado no \textit{brModelo} e implementado no \textit{HeidiSQL} para garantir a correta estruturação dos dados. A interação entre \textit{front-end} e \textit{back-end} será implementada usando \textit{Python} para a integração da visão computacional e \textit{Node.js} no computador, enquanto o \textit{React Native} será utilizado para criar uma API no \textit{mobile}, com foco na criação de uma API eficiente.

\subsection{Testes e validação}
Após a implementação, serão realizados testes automatizados e manuais para validar o funcionamento correto de cada parte do sistema. Ferramentas como Selenium poderão ser utilizadas para automatizar os testes de interface.

\subsection{Publicação e acompanhamento}
O sistema será implementado na plataforma Web, sendo feito com \textit{front-end} e \textit{back-end} primeiramente, e em uma versão futura, algumas das funções disponíveis na Web serão adaptadas para \textit{mobile}, com o objetivo inicial de permitir o uso do aplicativo no campo para escanear a folha em tempo real e receber o feedback instantâneo. A fácil escalabilidade e manutenção serão garantidas, e ajustes serão feitos com base no feedback dos usuários após a implantação.

\section*{RESULTADOS PRELIMINARES}\label{sect:resultados}

Após aplicar os métodos descritos, foram alcançados os seguintes resultados: a criação do site do projeto, utilizando HTML, CSS, JavaScript e NODE.JS, a criação do Diagrama de Classe e Diagrama de Objeto, além do banco de dados físico (MBD).

\textbf{Prototipação do aplicativo no Figma}

A \Cref{fig:landing-page} mostra a tela de Landing Page do projeto, onde são exibidas as principais funcionalidades.

\begin{figure}[H]
    \centering
    \caption{Landing Page}
    \includegraphics[width=0.7\linewidth]{Illustrations/Landing-Page.png}
    \label{fig:landing-page}
    \SourceOrNote{Autoria Própria (2024)}
\end{figure}

A tela inicial \Cref{fig:tela-inicio} oferece as opções de escanear uma folha com a câmera do celular ou fazer o upload de uma imagem para análise. Abaixo dessas opções, há um gráfico de pizza que mostra a porcentagem das ocorrências totais: amarelo para manganês, laranja-avermelhado para cobre e cinza para “adversos” (casos sem deficiência de cobre ou manganês). No lado direito, um quadro exibe análises recentes, incluindo o número total de plantas analisadas, tratadas e plantadas. No canto esquerdo, um menu permite acesso a outras telas do aplicativo, como histórico, drone e configurações gerais.

\begin{figure}[H]
\centering
\caption{Tela Início}
\includegraphics[width=0.7\linewidth]{Illustrations/tela-inicios.png}
\label{fig:tela-inicio}
\SourceOrNote{Autoria Própria (2024)}
\end{figure}

Ao selecionar “escanear folha”, o usuário deverá apontar a câmera para a folha escolhida. Caso opte por fazer o upload de uma imagem, o aplicativo iniciará a análise e, ao concluir, abrirá a tela \Cref{fig:cadastro-diagnóstico} mostrando a probabilidade da deficiência identificada. O usuário deve selecionar a planta analisada, o talhão ao qual pertence e pode adicionar um relatório, se necessário, antes de finalizar.

\begin{figure}[H]
\centering
\caption{Cadastro Diagnóstico}
\includegraphics[width=0.8\linewidth]{Illustrations/Tela-Escaneamentos.png}
\label{fig:cadastro-diagnóstico}
\SourceOrNote{Autoria Própria (2024)}
\end{figure}

Para verificar as plantas cadastradas no aplicativo, o usuário acessa a tela de histórico \Cref{fig:tela-relatorios}. Nessa tela, é possível ver a quantidade de plantas cadastradas em cada talhão, selecionar a propriedade sendo verificada, bem como visualizar o número de plantas analisadas até o momento. No lado direito, uma janela exibe o total de talhões e plantas registrados, além do número de plantas analisadas e diagnosticadas.

\begin{figure}[H]
\centering
\caption{Tela Relatórios}
\includegraphics[width=0.7\linewidth]{Illustrations/tela-historico.png}
\label{fig:tela-relatorios}
\SourceOrNote{Autoria Própria (2024)}
\end{figure}

Na tela de mapas \Cref{fig:tela-mapas}, o usuário pode escolher entre diferentes tipos de mapas, como o NDVI, um mapa via satélite que mostra a localização, um mapa com todas as mexericas e um mapa de calor que exibe a concentração de possíveis deficiências.

O NDVI, ou Índice de Estado de Vegetação, é gerado a partir de imagens capturadas por drones e um gráfico de linhas que mostra o nível de NDVI para cada talhão ao longo de 12 meses. Esse índice mede a quantidade de energia refletida e absorvida pelas plantas, fornecendo informações sobre sua saúde com base nessa refletância. A luz visível (400 a 720 nm) é absorvida, enquanto o infravermelho próximo (720 a 1100 nm) é refletido em maior intensidade por plantas saudáveis. Plantas em estresse, desidratadas ou doentes absorvem mais luz infravermelha, o que afeta seu índice NDVI. Esse valor é calculado usando a fórmula: \textit{NDVI} = \((\text{NIR} - \text{VIS}) / (\text{NIR} + \text{VIS})\) \cite{ResultadoNDVIArtigo, ResultadoNDVISite}.

\begin{figure}[H]
\centering
\caption{Mapas}
\includegraphics[width=0.7\linewidth]{Illustrations/tela-mapas.png}
\label{fig:tela-mapas}
\SourceOrNote{Autoria Própria (2024)}
\end{figure}

\textbf{Diagrama de classes}

O diagrama de classes do nosso projeto \Cref{fig:diagrama-classe} começa com a Landing Page, que é a tela inicial exibida caso o usuário não esteja logado. Nessa tela, o usuário pode optar por fazer login ou cadastrar-se. A tela de login permite que o usuário insira seu e-mail ou telefone celular e senha, com a opção de recuperação de senha, caso necessário.

O processo de cadastro envolve a inserção de dados como nome, sobrenome, e-mail, senha, telefone celular, confirmação de senha e endereço. Após essa etapa, o usuário pode continuar o cadastro, especificando o tipo de pessoa (física ou jurídica), CPF ou CNPJ (dependendo do tipo), logradouro, número, bairro e cidade. Em seguida, o sistema permite que o usuário cadastre sua propriedade, incluindo informações como nome, CEP, logradouro, número, bairro e cidade da propriedade. Esse cadastro de propriedade possibilita também a edição de talhões, permitindo ao usuário adicionar o nome do talhão e o nome do pé, com opções de edição ou finalização do cadastro.

Ao escolher escanear uma folha ou fazer o upload de uma imagem, o sistema realiza a análise da imagem e exibe uma tela de resultado que mostra a probabilidade de deficiência identificada ou outras condições. O usuário pode então informar o nome do pé analisado, o talhão ao qual ele pertence e, se necessário, adicionar um relatório à análise. O sistema permite editar o nome do pé, o talhão ou o relatório da análise antes de finalizar o cadastro do resultado.

Há também uma funcionalidade de histórico da propriedade, onde o usuário pode selecionar uma propriedade específica ou alternar para outro talhão, além de adicionar novos talhões. Ao escolher um talhão, o usuário consegue visualizar todos os pés cadastrados nele, podendo buscar um pé pelo nome, selecionar qualquer um dos pés, editar o talhão ou adicionar novos pés. Caso um pé específico seja selecionado, o usuário pode editar o nome, o relatório e, se necessário, excluí-lo do sistema.

\begin{figure}[H]
\centering
\caption{Diagrama de Classe}
\includegraphics[width=0.7\linewidth]{Illustrations/diagramaClasse.png}
\label{fig:diagrama-classe}
\SourceOrNote{Autoria Própria (2024)}
\end{figure}



\section*{CONCLUSÃO}\label{sect:conclusao}

Inicialmente, o projeto tem por objetivo buscar uma solução para as deficiências de minerais encontradas na mexeriqueira (Citrus reticulata), especificamente na variedade conhecida como mexerica, e, consequentemente, apresentar uma alternativa ao problema utilizando tecnologia. Dessa forma, elaboramos um projeto cuja ideia central é criar uma aplicação capaz de identificar, em tempo real, a deficiência de minerais pela qual a planta está passando. Isso permitirá produzir um diagnóstico preciso, facilitando o processo de recuperação da planta, evitando a perda de frutos e otimizando o tempo dos produtores.

Diante disso, nos deparamos com algumas limitações ao longo da pesquisa, pois ela é restrita a apenas alguns países e às condições do solo. A falta de minerais na planta possui relação direta com a acidez do solo. Nosso estudo tem por objetivo identificar especificamente deficiências de manganês e cobre em plantas de Citrus reticulata (mexerica) através da análise foliar. Pesquisas mostram que uma planta com deficiência de manganês geralmente não apresenta deficiência de cobre e vice-versa. Uma análise mais aprofundada revela que a deficiência está diretamente ligada ao pH do solo. A carência de manganês pode ser identificada em um solo mais ácido, enquanto a deficiência de cobre pode se manifestar em um solo alcalino \cite{ConclusãoMicroN, ConclusãoCobre}.

\printbibliography

%% Elementos pós-textuais (opcionais): Apêndice e Anexo
%Caso for utilizar, basta retirar o símbolo de % na frente do comando
%%%%% Elementos pós-textuais
%%
%% Glossário, apêndices, anexos e índice remissivo (opcionais).

%% Apêndices
\begin{Appendix}

    \section{Modelo de negócios Canvas}%
    \label{sect:apx-a1}
    
    \begin{figure}[H]
    \centering
    \caption{Modelo de negócios Canvas}%
    \label{fig:canvaspi}
    \includegraphics[width=0.8\linewidth]{Illustrations/canvas.png}
    \SourceOrNote{Autoria Própria (2024)}
    \end{figure}
    
\end{Appendix}
    
    
    %% Índice remissivo
\printindex%
    

%%%% Elementos pós-textuais
%%
%% Glossário, apêndices, anexos e índice remissivo (opcionais).

%% Apêndices
\begin{Appendix}

    \section{Modelo de negócios Canvas}%
    \label{sect:apx-a1}
    
    \begin{figure}[H]
    \centering
    \caption{Modelo de negócios Canvas}%
    \label{fig:canvaspi}
    \includegraphics[width=0.8\linewidth]{Illustrations/canvas.png}
    \SourceOrNote{Autoria Própria (2024)}
    \end{figure}
    
\end{Appendix}
    
    
    %% Índice remissivo
\printindex%
    

%% Fim do documento
\end{document}