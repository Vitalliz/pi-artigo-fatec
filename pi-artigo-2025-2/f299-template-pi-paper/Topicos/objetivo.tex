A citricultura enfrenta desafios constantes relacionados a doenças e deficiências nutricionais que comprometem o desenvolvimento saudável das plantas e reduzem a produtividade. Embora essas condições possam ser identificadas visualmente por alterações nas folhas, cascas e frutos, o diagnóstico tradicional — baseado em análises químicas — é demorado, exige coleta física e depende de infraestrutura laboratorial especializada. Nesse cenário, o uso da IA aplicada à visão computacional apresenta-se como uma alternativa acessível e eficiente, permitindo diagnósticos rápidos e automáticos diretamente no campo.

Diante desse contexto, o presente projeto tem como objetivo geral desenvolver um sistema inteligente, baseado em visão computacional, capaz de identificar deficiências de manganês e cobre em folhas de mexerica por meio da análise de imagens capturadas com dispositivos móveis. A proposta busca oferecer maior agilidade e precisão no diagnóstico nutricional, contribuindo para a sustentabilidade da produção e a redução de perdas causadas por deficiências minerais.

\textbf{Objetivos Específicos:}
\begin{enumerate}
\item Construir um banco de dados com imagens rotuladas de folhas de mexerica apresentando diferentes níveis de deficiência de manganês e cobre, de modo a compor uma base de dados representativa para o treinamento da IA.
\item Treinar e validar uma CNN utilizando técnicas de aumento de dados, validação cruzada e ajuste de hiperparâmetros, com o objetivo de otimizar a acurácia do modelo.
\item Implementar um protótipo funcional que permita ao agricultor obter diagnósticos automatizados, por meio da captura de imagens via smartphone, com base na análise da IA.
\end{enumerate}

\textbf{Resultados Esperados:}

Com o desenvolvimento do sistema, espera-se disponibilizar uma ferramenta prática e acessível para pequenos e médios produtores, reduzindo o tempo e o custo do diagnóstico nutricional. A solução visa auxiliar na detecção precoce de deficiências minerais, promover o uso racional de insumos e fortalecer práticas agrícolas mais sustentáveis e eficientes.

\textbf{Trabalhos Futuros:}

Futuras etapas do projeto poderão expandir a aplicação do sistema inteligente para além do diagnóstico em folhas de mexerica. Entre as possibilidades, destacam-se o uso de drones para captura de imagens em grande escala, a geração de mapas de calor indicando áreas com deficiências nutricionais na plantação e a adaptação do modelo para outras culturas cítricas ou frutíferas, ampliando o alcance e a utilidade da ferramenta para diferentes contextos agrícolas.