A citricultura enfrenta desafios constantes relacionados a doenças e deficiências nutricionais que comprometem o desenvolvimento saudável das plantas e reduzem a produtividade. Embora essas condições possam ser identificadas visualmente por alterações nas folhas, cascas e frutos, o diagnóstico tradicional — baseado em análises químicas — é demorado, exige coleta física e depende de infraestrutura laboratorial especializada. Nesse cenário, o uso da IA aplicada à visão computacional apresenta-se como uma alternativa acessível e eficiente, permitindo diagnósticos rápidos e automáticos diretamente no campo.

O projeto tem como objetivo principal desenvolver um sistema inteligente baseado em visão computacional para identificar deficiências de manganês e cobre em folhas de Citrus reticulata (mexerica), por meio da análise de imagens capturadas com dispositivos móveis. O sistema visa proporcionar agilidade, precisão e autonomia ao agricultor, contribuindo para a sustentabilidade da produção e a redução de perdas.

\textbf{Objetivos Específicos:}
\begin{enumerate}
\item Construir um banco de dados com imagens rotuladas de folhas de mexerica apresentando diferentes níveis de deficiência de manganês e cobre, de modo a compor uma base de dados representativa para o treinamento da IA.
\item Treinar e validar uma CNN utilizando técnicas de aumento de dados, validação cruzada e ajuste de hiperparâmetros, com o objetivo de otimizar a acurácia do modelo.
\item Implementar um protótipo funcional que permita ao agricultor obter diagnósticos automatizados, por meio da captura de imagens via smartphone, com base na análise da IA.
\end{enumerate}

\textbf{Além disso, o projeto também contempla os seguintes objetivos específicos:}
\begin{enumerate}
\item Habilitar o registro dos diagnósticos no sistema, permitindo ao usuário cadastrar o número da planta e o talhão, facilitando o acompanhamento e controle nutricional.
\item Implementar uma funcionalidade de histórico, permitindo ao agricultor visualizar a evolução do estado das plantas e compará-lo com registros anteriores.
\item Criar um módulo de recomendações técnicas, com sugestões agronômicas baseadas nos resultados obtidos, visando a aplicação mais assertiva de insumos.
\item Prever, para fases futuras, o uso de drones para captura de imagens aéreas, com o objetivo de identificar áreas críticas da plantação.
\item Incluir, posteriormente, funcionalidades como mapas interativos e mapas de calor, que indiquem visualmente a distribuição das deficiências por talhão, facilitando a tomada de decisão.
\item Explorar a possibilidade de expansão da plataforma para outras culturas e deficiências nutricionais, visando aumentar o escopo e escalabilidade da solução.
\item Contribuir para a sustentabilidade agrícola por meio da redução de perdas por deficiência nutricional e aumento da rentabilidade dos produtores rurais por meio do uso direcionado de insumos.
\end{enumerate}