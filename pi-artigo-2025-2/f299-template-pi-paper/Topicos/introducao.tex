A escassez de alimentos continua sendo um desafio global, agravado por fatores como pobreza, conflitos e mudanças climáticas. Em resposta, a Organização das Nações Unidas (ONU) instituiu, em 2015, os Objetivos de Desenvolvimento Sustentável (ODS). Entre eles, o ODS 2 — Fome Zero e Agricultura Sustentável — busca assegurar a segurança alimentar por meio do aumento da produtividade agrícola e do uso de tecnologias sustentáveis.

No Brasil, um dos principais entraves à produtividade agrícola está relacionado às deficiências nutricionais nas plantas, especialmente em culturas cítricas como a \textit{Citrus reticulata} (mexerica), que apresentam elevada sensibilidade a desequilíbrios minerais. Essas deficiências afetam diretamente o desenvolvimento vegetativo e o rendimento da produção, refletindo-se em perdas econômicas para pequenos e médios produtores. Estudos recentes indicam que a concentração de nutrientes em plantas cítricas varia sazonalmente no solo, nas folhas e na seiva do xilema, o que dificulta o diagnóstico preciso do estado nutricional e contribui para manejos inadequados de adubação e correção do solo \cite{FaveroFilho2022}. 

Durante a pesquisa de campo realizada em propriedades rurais produtoras de mexerica, foi observada alta incidência de deficiência de manganês e baixa de cobre, além de numerosos casos de greening (HLB) distribuídos por toda a plantação. Embora o foco deste estudo não seja essa doença, sua menção é relevante, pois os sintomas visuais — como clorose e deformações foliares — podem ser confundidos com deficiências nutricionais, o que reforça a importância de diagnósticos precisos e acessíveis \cite{Fundecitrus2021, Aregbe2024}.

A carência de manganês manifesta-se por clorose internerval em folhas jovens, enquanto a deficiência de cobre provoca encurtamento dos ramos, folhas pequenas e ocorrência de gomose nos frutos. Essas condições afetam diretamente a produtividade e estão associadas ao pH do solo — sendo o manganês mais escasso em solos ácidos e o cobre em solos mais alcalinos \cite{Bruna2019, Machado2022}.

Durante a última década, técnicas de Inteligência Artificial (IA) e \textit{deep learning} têm sido cada vez mais aplicadas na agricultura de precisão, principalmente no diagnóstico de deficiências nutricionais em plantas. Entre os métodos de aprendizado de máquina, as redes neurais convolucionais (CNNs), uma das principais arquiteturas de \textit{deep learning}, destacam-se por sua capacidade de extrair automaticamente características visuais complexas de imagens, sem a necessidade de pré-processamento extensivo ou seleção manual de atributos. Diferentemente de algoritmos clássicos, como SVMs, árvores de decisão ou k‑NN, que dependem fortemente de variáveis pré-definidas e da engenharia de características, as CNNs conseguem aprender representações hierárquicas diretamente a partir das imagens das folhas, capturando padrões sutis de clorose, deformações ou manchas associadas a deficiências nutricionais \cite{Qin2018, Christin2019}.

Ao integrar tais tecnologias ao contexto agrícola brasileiro, o projeto busca contribuir com os objetivos do ODS 2 — Fome Zero e Agricultura Sustentável — promovendo práticas mais sustentáveis, reduzindo perdas na produção e fortalecendo a segurança alimentar por meio da inovação tecnológica.