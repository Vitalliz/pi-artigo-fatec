A pesquisa científica tem se voltado para o desenvolvimento de tecnologias que auxiliem o diagnóstico rápido e preciso de deficiências nutricionais em plantas, especialmente em culturas de relevância econômica, como a \textit{Citrus reticulata} (mexerica). A identificação visual dessas deficiências é um desafio recorrente, e soluções baseadas em visão computacional e \textit{deep learning} têm se mostrado promissoras por permitirem análises automatizadas e de baixo custo. A seguir, são apresentados três trabalhos científicos que exploram essas abordagens e servem como referência para o desenvolvimento do projeto NitrusLeaf.

No trabalho de \textcite{Muthusamy2023}, foi proposto um sistema de monitoramento agrícola baseado em visão computacional, \textit{machine learning} e \textit{deep learning}. O estudo aplicou essas técnicas em diferentes culturas agrícolas, utilizando imagens de satélite, sensoriamento remoto, \textit{Internet of Things} (IoT) e Veículos Aéreos Não Tripulados (VANTs) para capturar dados visuais em tempo real e identificar deficiências nutricionais. O método integrou sensores e câmeras de alta resolução que coletaram imagens sob variadas condições ambientais, processadas por algoritmos de IA capazes de identificar padrões visuais como cor, textura e bordas. Os resultados mostraram diagnósticos rápidos e não invasivos, otimizando o manejo agrícola e o uso de insumos. Como limitação, os autores apontaram o alto custo dos equipamentos e a dificuldade de aplicação em pequenas propriedades. Esse estudo fornece base teórica relevante para o NitrusLeaf, especialmente quanto ao uso de algoritmos de visão computacional, embora o projeto atual priorize a acessibilidade e o uso direto em campo com dispositivos móveis.

\textcite{Ghorai2021} desenvolveram um \textit{pipeline} para detecção automatizada de doenças e deficiências nutricionais em plantas, utilizando processamento digital de imagens. A metodologia foi estruturada em seis etapas: aquisição da imagem, pré-processamento, segmentação, extração de características, classificação e diagnóstico. As imagens foram capturadas por câmeras acopladas a drones e dispositivos móveis e, posteriormente, padronizadas com técnicas de correção de ruído e ajuste de contraste. A análise das características visuais envolveu cor, textura e forma, destacando o uso da Matriz de Coocorrência de Níveis de Cinza (GLCM) e histogramas de cor. A classificação foi realizada com redes neurais convolucionais (CNNs) e Máquinas de Vetores de Suporte (SVMs), resultando em acurácia superior a 90\%. Apesar do bom desempenho, os autores destacaram limitações quanto à generalização dos modelos em diferentes espécies, devido à necessidade de bases de dados mais diversificadas. A abordagem é relevante para o NitrusLeaf por oferecer uma estrutura metodológica aplicável à análise de folhas de mexerica, permitindo a detecção precisa das deficiências de cobre e manganês.

\textcite{Tran2019} aplicaram redes neurais convolucionais para detectar deficiências de cálcio, nitrogênio e potássio em tomateiros, utilizando modelos Inception-ResNet v2, Autoencoder e uma combinação híbrida por \textit{ensemble averaging}. O conjunto de dados foi composto por 571 imagens coletadas em estufas, sendo 80\% destinadas ao treinamento e 20\% à validação. Foram aplicadas técnicas de pré-processamento e \textit{data augmentation}, como variação de ângulo, brilho e contraste, para aprimorar a generalização dos modelos. Os resultados mostraram acurácia de 87,27\% para o Inception-ResNet v2, 79,09\% para o Autoencoder e 91\% para o \textit{ensemble}. O estudo demonstrou que abordagens híbridas aumentam a precisão dos diagnósticos, embora exijam alto custo computacional e recursos de hardware avançados. Para o NitrusLeaf, os resultados de \textcite{Tran2019} reforçam a importância de arquiteturas otimizadas que equilibrem desempenho e eficiência, considerando o uso em dispositivos móveis e ambientes agrícolas.

A análise comparativa dos três trabalhos evidencia abordagens complementares e contribuições significativas. \textcite{Muthusamy2023} propõem uma aplicação em larga escala com integração de tecnologias inteligentes; \textcite{Ghorai2021} apresentam uma metodologia acessível e bem estruturada baseada em etapas de processamento de imagem; e \textcite{Tran2019} demonstram o potencial das redes híbridas para diagnósticos de alta precisão. O NitrusLeaf combina elementos dessas pesquisas, adaptando-os à realidade da mexerica, com foco na detecção de deficiências de cobre e manganês por meio de IA aplicada a dispositivos móveis. A proposta busca unir precisão técnica, baixo custo operacional e aplicabilidade prática, atendendo especialmente às necessidades de pequenos e médios produtores rurais.