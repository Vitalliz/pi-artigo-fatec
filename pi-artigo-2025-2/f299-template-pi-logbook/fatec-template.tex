\documentclass[
    landscape,
    a4paper,
    12pt,
    english,
    brazilian,
]{article}

\usepackage{fatec-article}
\usepackage{array} % Necessário para 'm{width}' para centralização vertical
\usepackage{longtable} % Necessário para tabelas que ocupam várias páginas
\usepackage{setspace}

\begin{document}

\section*{Diário de Bordo}

% Usando o ambiente longtable para tabelas que se estendem por várias páginas
\begin{longtable}{|m{4cm}|m{2.8cm}|m{2.8cm}|m{4.8cm}|m{8cm}|} % Usando 'm' para centralizar verticalmente
    \hline
    Nome da Atividade & Data de início & Data de término & Responsável pela atividade & Descrição da atividade realizada \\ \hline
    \endfirsthead
    % Cabeçalho que aparecerá no início de cada página
    \hline
    Nome da Atividade & Data de início & Data de término & Responsável pela atividade & Descrição da atividade realizada \\ \hline
    \endhead
    
    % Aqui você adiciona as linhas da tabela
    \centering Formulários e Autorização de Pesquisa de Campo & \centering 26/08/2025 & \centering 16/09/2025 & \centering Amanda, Valéria & Realizado pesquisa e formulação de perguntas para o questionário e alterações do corpo da autorização para a pesquisa acadêmico-científica do projeto integrador. \\ \hline
    \centering Briefing & \centering 17/09/2025 & \centering 29/09/2025 & \centering Amanda, Lucas, Valéria & Todos os integrantes realizaram em partes a criação de personas, roteiro de briefing e slides de apresentação do Briefing do Projeto Integrador. \\ \hline
    \centering Pesquisa de Campo & \centering 20/09/2025 & \centering 20/09/2025 & \centering Amanda, Lucas & Realizado visitas ao Sítio São Miguel (empresa Frutas Wosniak) e à Fazenda Eizo Rio Rainha situadas em Pariquera-Açu, sendo estes produtores da cultura estudada. Foi apresentado o conceito do projeto e feito a coleta das respostas do questionário. \\ \hline
    \centering Prototipação no Figma & \centering 21/09/2025 & \centering 28/09/2025 & \centering Amanda & Atualizações nas telas na adição de tela de cadastro e inclusão de tutorial no protótipo da aplicação. \\ \hline
    \centering Artigo e Artefatos & \centering 02/10/2025 & \centering 27/10/2025& \centering Lucas, Valéria & Desenvolvimento da seção de Metodologias e atualização de Artefatos. \\ \hline
    \centering Aplicação Mobile & \centering 16/10/2025 & \centering 29/10/2025& \centering Amanda, Valéria & Criado repositório de projeto em versão mobile, desenvolvido em React.js. \\ \hline
    \centering Slides de Apresentação & \centering 18/10/2025 & \centering 29/10/2025 & \centering Amanda & Feito criação de slides de apresentação com o template da CPS em Latex Beamer. \\ \hline
    \centering Banner & \centering 24/10/2025  & \centering 24/10/2025 & \centering Amanda & Atualização do banner que será utilizado para a próxima feira tecnológica. \\ \hline
    \centering Banco de Dados & \centering 24/10/2025 & \centering 25/10/2025& \centering Lucas & O banco de dados do site foi implementado e foi atualizado. Está alinhado às funcionalidades previstas para o projeto. \\ \hline
    \centering Diagrama de Casos de Uso & \centering 24/10/2025 & \centering 25/10/2025& \centering Lucas & Atualizado o diagrama de caso de uso, com todos os atores e suas respectivas ações corretas e alinhadas com o projeto atual. \\ \hline
    \centering Diagrama de Redes & \centering 25/10/2025 & \centering 25/10/2025& \centering Lucas & Finalizamos o diagrama de redes do NitrusLeaf: o app envia cadastros e imagens via HTTPS para uma API em Node.js, que orquestra um microserviço Python (CNN) para classificação nutricional e persiste os resultados em MySQL na nuvem. Implantamos um painel administrativo para observabilidade e gestão do modelo/BD. \\ \hline
    \centering Pitch & \centering 25/10/2025 & \centering 28/10/2025& \centering Lucas & Criação de novo pitch com vídeos dos integrantes em inglês, com legendas em Português e Inglês. \\ \hline

    % Continue adicionando linhas conforme necessário
    
\end{longtable}  

\end{document}
