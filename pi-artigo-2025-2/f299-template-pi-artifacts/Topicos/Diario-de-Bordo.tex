Durante os três semestres de desenvolvimento anteriores, realizamos diversas
atividades que contribuíram para a construção e evolução do projeto. Abaixo,
apresentamos imagens detalhadas sobre cada etapa durante os semestres:

% -- 1° Semestre --
\begin{center}
\captionof{table}{Diário de Bordo – 2º Semestre}
\begin{adjustbox}{max width=\textwidth}
\begin{tabular}{|m{4cm}|m{2.5cm}|m{2.5cm}|m{3cm}|m{8cm}|}
\hline
\textbf{Nome da Atividade} &
\textbf{Data de Início} &
\textbf{Data de Término} &
\textbf{Responsável pela Atividade} &
\textbf{Descrição da Atividade Realizada} \\ \hline

Pesquisa temas para o Projeto & 27/02/2024 & 09/03/2024 & Atividade realizada em grupo &
Realizamos pesquisas para possíveis temas, resultando como escolhido o tema Identificação de deficiência de manganês e cobre na folha da mexerica. \\ \hline

Pesquisas de artigos científicos & 10/03/2024 & 16/03/2024 & Atividade realizada em grupo &
Realizamos pesquisas de artigos científicos para fortalecer o desenvolvimento do tema. \\ \hline

Elaboração do artigo & 17/03/2024 & 19/03/2024 & Atividade realizada em grupo &
Cada um ficou responsável por desenvolver um tópico do artigo se baseando nas pesquisas e reuniões em grupo. \\ \hline

Introdução & 20/03/2024 & 18/04/2024 & Luiz &
Realização da introdução do artigo científico. \\ \hline

Objetivo & 20/03/2024 & 18/04/2024 & Amanda Vithória &
Realização do tópico objetivo do artigo científico. \\ \hline

Estado da Arte & 20/03/2024 & 18/04/2024 & Lucas &
Realização do Estado da Arte do artigo científico. \\ \hline

Metodologia & 20/03/2024 & 18/04/2024 & Valéria &
Realização da metodologia do artigo científico. \\ \hline

Criação do manual de identidade visual & 31/03/2024 & 21/04/2024 & Atividade realizada em grupo &
Como proposto na aula de design, deveríamos elaborar um manual para representar a identidade visual do nosso projeto e da nossa equipe. \\ \hline

Criação das logos & 31/03/2024 & 05/04/2024 & Lucas e Amanda Vithória &
Elaborar as logos da equipe e do projeto. \\ \hline

Escolha das paletas de cores & 06/04/2024 & 08/04/2024 & Decisão tomada em grupo &
Escolher as cores que estarão no projeto. \\ \hline

Escolha da tipografia & 08/04/2024 & 10/04/2024 & Decisão tomada em grupo &
Escolher as tipografias que estarão no projeto. \\ \hline

Apresentação do manual de identidade visual & 22/04/2024 & 22/04/2024 & -- &
Apresentação do manual de identidade visual. \\ \hline

Modelo de baixa fidelidade do Figma & 23/04/2024 & 26/04/2024 & Valéria &
Estruturamos o modelo da aplicação que mostra quais telas são necessárias e quais elementos são importantes para seu funcionamento, norteando o design e auxiliando no banco de dados do projeto. \\ \hline

Modelo conceitual de banco de dados & 25/04/2024 & 30/04/2024 & Valéria &
O modelo conceitual é responsável por definir entidades e o relacionamento entre elas, norteando como o sistema deve funcionar. \\ \hline

Modelo de Alta Fidelidade & 28/04/2024 & 01/06/2024 & Amanda Vithória &
Estruturar o modelo da aplicação que mostra com detalhes como cada tela vai funcionar. Nesse processo, as telas irão servir como uma prévia final de aplicação, demonstrando a interação do usuário com o sistema. \\ \hline

Oracle APEX do projeto & 06/05/2024 & 01/06/2024 & Luiz &
Responsável por mostrar um site que exibe as telas previstas no projeto mais desenvolvido, incluindo mapa, mapa de calor e gráficos com quantidade de incidências e não incidências. \\ \hline

Diagrama de Redes & 13/05/2024 & 20/05/2024 & Lucas e Luiz &
O diagrama de redes mostra como a infraestrutura do projeto irá funcionar. \\ \hline

Diagrama de Caso de Uso & 13/05/2024 & 20/05/2024 & Amanda e Luiz &
O diagrama de caso de uso mostra os processos que ocorrem durante a utilização do software. \\ \hline

Site da Equipe & 13/05/2024 & 27/05/2024 & Lucas &
O site descreve um pouco da equipe, mostrando o nicho de atuação dos integrantes e uma breve descrição do projeto. \\ \hline

Banner & 20/05/2024 & 03/06/2024 & Amanda Vithória &
O banner demonstra de forma resumida todo o projeto para apresentações ao público. \\ \hline

\end{tabular}
\end{adjustbox}

\vspace{0.3em}
\small{\textbf{Fonte:} Equipe 21 – Vitalliz (2024)}

\end{center}

% -- 2° Semestre --
\begin{center}
\captionof{table}{Diário de Bordo – 2º Semestre}
\begin{adjustbox}{max width=\textwidth}
\begin{tabular}{|m{4cm}|m{2.2cm}|m{2.2cm}|m{3cm}|m{8cm}|}
\hline
\textbf{Nome da Atividade} &
\textbf{Data de Início} &
\textbf{Data de Término} &
\textbf{Responsável pela Atividade} &
\textbf{Descrição da Atividade Realizada} \\ 
\hline

Revisão do Artigo & 08/10/2024 & 17/11/2024 & Luiz &
Foram corrigidos erros de português, revisados os objetivos e reformulada a seção de estado da arte. Além disso, incluíram-se os resultados preliminares com as telas do site, o modelo físico do banco de dados e explicações sobre o diagrama de classes e objetos na seção de resultados. Por fim, algoritmos de recursividade foram implementados na tela de busca do site. \\ \hline

Prototipação das Telas do Site no Figma & 05/10/2024 & 17/11/2024 & Amanda &
Desenvolvimento das telas iniciais com base na prototipação do semestre anterior, incluindo as telas da versão mobile, com adição de landing page e reformulação do design anterior para a versão desktop web. Foram criados componentes para reduzir o número de telas e tornar o desenvolvimento mais eficiente, além de aprimorar o design para facilitar a visualização da simulação. \\ \hline

Desenvolvimento da Parte Front-End do Site & 15/10/2024 & 17/11/2024 & Amanda e Valéria &
Criação das telas web seguindo fielmente as telas prototipadas no Figma. O desenvolvimento do design foi feito utilizando CSS e Bootstrap, mantendo o layout responsivo e visualmente coerente com o projeto. \\ \hline

Desenvolvimento da Parte Back-End do Site & 15/10/2024 & 17/11/2024 & Todos os integrantes do grupo &
Foram implementadas as funcionalidades do protótipo criado no Figma, incluindo sistemas internos de rota e integração com o banco de dados. Foram utilizadas bibliotecas e funcionalidades do Node.js, como Express, Nodemon, Middleware, View Engine EJS e Sequelize (para integração com MySQL2). A arquitetura MVC foi adotada para estruturar melhor as pastas e organizar o código. \\ \hline

Banco de Dados Físico & 25/10/2024 & 10/11/2024 & Lucas &
Criação do banco de dados utilizado no site, estruturado conforme todas as funcionalidades previstas no projeto, garantindo coerência entre o modelo físico e os requisitos do sistema. \\ \hline

Diagrama de Entidade-Relacionamento (DER) & 23/10/2024 & 25/10/2024 & Lucas &
O DER do banco de dados foi refeito, incluindo todos os dados corretos e alinhados com a versão atual do projeto, garantindo consistência e completude no modelo conceitual. \\ \hline

Diagrama de Classe & 25/09/2024 & 17/11/2024 & Luiz &
Elaboração do diagrama de classes do projeto, com base nas funções implementadas. O diagrama representa a estrutura de classes e suas relações, definindo atributos e métodos de cada componente do sistema. \\ \hline

Diagrama de Objetos & 25/09/2024 & 17/11/2024 & Luiz &
Criação do diagrama de objetos, representando instâncias concretas das classes principais do projeto e demonstrando as interações entre elas. \\ \hline

Diagrama de Caso de Uso & 11/11/2024 & 17/11/2024 & Valéria &
Refação do diagrama de caso de uso, contemplando todos os atores e suas respectivas ações, alinhadas com as funcionalidades atuais do sistema. \\ \hline

Banner & 13/11/2024 & 18/11/2024 & Amanda &
Desenvolvimento do banner do projeto, que será utilizado na próxima feira tecnológica, com foco na identidade visual e clareza na apresentação das informações. \\ \hline

\end{tabular}
\end{adjustbox}

\vspace{0.3em}
\small{\textbf{Fonte:} Equipe 21 – Vitalliz (2024)}

\end{center}

% -- 3° Semestre --
\begin{center}
\captionof{table}{Diário de Bordo – 3º Semestre}
\begin{adjustbox}{max width=\textwidth}
\begin{tabular}{|m{4cm}|m{2.2cm}|m{2.2cm}|m{3cm}|m{8cm}|}
\hline
\textbf{Nome da Atividade} &
\textbf{Data de Início} &
\textbf{Data de Término} &
\textbf{Responsável pela Atividade} &
\textbf{Descrição da Atividade Realizada} \\ 
\hline

Diagrama de Banco de Dados Conceitual (DER) & 18/02/2025 & 28/02/2025 & Luiz &
Refação completa do banco de dados, iniciando pelo Diagrama Entidade-Relacionamento (DER). O banco foi reestruturado para alinhar com as exigências e requisitos do projeto, otimizando o armazenamento de dados. \\ \hline

Diagrama de Banco de Dados Lógico (MER) & 09/03/2025 & 10/03/2025 & Valéria &
Elaboração do Modelo Entidade-Relacionamento Lógico (MER), baseado no DER. A estrutura foi ajustada para garantir a integridade e eficiência do banco de dados, atendendo aos requisitos do sistema. \\ \hline

Modelo Físico do Banco & 18/02/2025 & 28/02/2025 & Luiz &
Desenvolvimento do Modelo Físico do Banco de Dados, aplicando as definições do DER e MER. A modelagem física define os tipos de dados e as tabelas de armazenamento para otimizar a consulta e performance do sistema. \\ \hline

Revisão do Artigo & 31/03/2025 & 11/05/2025 & Luiz &
Revisão do artigo conforme o feedback recebido na última banca. A introdução foi reescrita para ser mais concisa, evitando redundâncias e abordando de maneira mais objetiva os pontos principais. O objetivo foi aprimorado de acordo com as orientações dos professores, melhorando sua clareza e alinhamento com o escopo do projeto. Pequenos ajustes foram feitos na metodologia e no estado da arte. \\ \hline

Análise SWOT & 20/04/2025 & 11/05/2025 & Valéria &
Realização da análise SWOT para identificar as forças, fraquezas, oportunidades e ameaças do projeto. A análise foi conduzida para melhor compreender os pontos fortes e fracos da solução proposta, além de mapear as oportunidades que podem ser aproveitadas e as ameaças que precisam ser mitigadas. Esse processo ajudou a ajustar o planejamento estratégico, proporcionando uma visão mais clara dos desafios e das vantagens competitivas do projeto. \\ \hline

Scrum & 20/04/2025 & 11/05/2025 & Valéria &
Aplicação do framework Scrum para organizar o projeto em sprints, com reuniões de planejamento, acompanhamento e revisão, garantindo agilidade e melhor controle das entregas. \\ \hline

Revisão do Pitch & 09/04/2025 & 19/04/2025 & Lucas &
Revisão do pitch de apresentação, incluindo a tradução das legendas para o inglês, a fim de ampliar a acessibilidade e alcançar um público internacional. Também foi feita a atualização das telas, substituindo as anteriores pelas versões mais recentes do sistema, refletindo o progresso atual do projeto. \\ \hline

Revisão da Logo da Equipe & 17/04/2025 & 30/04/2025 & Amanda &
A logo da equipe foi revista e reformulada para melhorar sua estética visual. A nova versão busca uma representação mais moderna e atrativa, alinhada com a identidade do projeto e com a proposta de inovação tecnológica. Além de melhorar a aparência, o design foi ajustado para garantir maior clareza e legibilidade, mantendo a consistência com os valores e objetivos do projeto. \\ \hline

Revisão do Site da Equipe & 01/03/2025 & 09/03/2025 & Lucas &
O site da equipe foi revisado para ser mais responsivo e estilizado. As mudanças incluíram ajustes no layout para garantir que o site se adaptasse a diferentes dispositivos, como celulares e tablets, oferecendo uma melhor experiência de usuário. Além disso, o design foi aprimorado com elementos visuais mais modernos, garantindo uma aparência mais profissional e alinhada à identidade do projeto. \\ \hline

Site do Projeto (Front-End) & 20/04/2025 & 11/05/2025 & Luiz, Amanda, Valéria, Lucas &
O site do projeto foi refeito, com foco no desenvolvimento do front-end utilizando HTML, CSS, JavaScript e React. O objetivo é garantir que o site seja responsivo, adaptando-se a diferentes dispositivos e proporcionando uma experiência de navegação fluida e intuitiva. O design está sendo construído com base nos princípios de Interação Humano-Computador (IHC), para melhorar a usabilidade (UI) e a experiência do usuário (UX), oferecendo uma interface moderna, clara e de fácil navegação. \\ \hline

Back-End do Projeto & 09/05/2025 & 12/05/2025 & Luiz &
O desenvolvimento do back-end do projeto está sendo feito utilizando Node.js em conjunto com o banco de dados MySQL. Foram implementadas funcionalidades essenciais como login, cadastro de usuários, envio de informações para o banco de dados e consultas de dados. A estrutura do back-end foi desenvolvida como uma API REST, desacoplada do front-end, garantindo maior flexibilidade e escalabilidade. Essa abordagem permite que o front-end se comunique com o back-end de forma eficiente e independente, assegurando uma melhor organização e modularidade no código. \\ \hline

Artefatos & 08/05/2025 & 11/05/2025 & Lucas &
Criação dos artefatos do projeto, com base nas informações solicitadas no mind-map. Foram organizados e documentados todos os avanços e entregas realizadas até o momento, incluindo a descrição detalhada dos componentes do sistema, processos e ferramentas utilizadas. Esses artefatos têm como objetivo fornecer uma visão clara e estruturada do projeto, alinhada às etapas já completadas, facilitando a comunicação entre a equipe e a documentação do progresso. \\ \hline

Banner & 07/05/2025 & 10/05/2025 & Amanda &
Criação do banner do projeto, que será utilizado na próxima feira de tecnologia da Fatec. \\ \hline

\end{tabular}
\end{adjustbox}

\vspace{0.3em}
\small{\textbf{Fonte:} Equipe 21 – Vitalliz (2025)}

\end{center}

% -- 4° Semestre --
\begin{center}

\captionof{table}{Diário de Bordo 4° Semetsre}

\begin{adjustbox}{max width=\textwidth}
\begin{tabular}{|m{3.5cm}|m{2.2cm}|m{2.2cm}|m{3.5cm}|m{6.5cm}|}
\hline
\textbf{Nome da Atividade} & 
\textbf{Data de Início} & 
\textbf{Data de Término} & 
\textbf{Responsável pela Atividade} & 
\textbf{Descrição da Atividade Realizada} \\ 
\hline

Formulários e Autorização de Pesquisa de Campo & 26/08/2025 & 16/09/2025 & Amanda, Valéria & Realizada pesquisa e formulação de perguntas para o questionário e alterações no corpo da autorização para pesquisa acadêmico-científica do projeto integrador. \\ \hline

Briefing & 17/09/2025 & 29/09/2025 & Amanda, Lucas, Valéria & Criação de personas, roteiro e slides para o briefing do Projeto Integrador. \\ \hline

Pesquisa de Campo & 20/09/2025 & 20/09/2025 & Amanda, Lucas & Visitas ao Sítio São Miguel (Frutas Wosniak) e Fazenda Eizo Rio Rainha em Pariquera-Açu. Coleta de respostas do questionário. \\ \hline

Prototipação no Figma & 26/09/2025 & – & Amanda & Criação de Wireframe e Wireflows do projeto. Adição de tela de cadastro e tutorial no protótipo. \\ \hline

Artigo e Artefatos & 02/10/2025 & – & Lucas, Valéria & Desenvolvimento da seção de Metodologia e atualização dos artefatos. \\ \hline

Aplicação Mobile & 16/10/2025 & – & Amanda, Valéria & Criação do repositório e desenvolvimento da versão mobile em React.js. \\ \hline

Slides de Apresentação & 18/10/2025 & – & Amanda & Criação de slides com o template CPS em LaTeX Beamer. \\ \hline

Banner & 24/10/2025 & 24/10/2025 & Amanda & Atualização do banner para a feira tecnológica. \\ \hline

Banco de Dados & 24/10/2025 & 25/10/2025 & Lucas & Implementação e atualização do banco de dados conforme funcionalidades do projeto. \\ \hline

Diagrama de Casos de Uso & 24/10/2025 & 25/10/2025 & Lucas & Atualização do diagrama com atores e ações alinhadas ao projeto. \\ \hline

Diagrama de Redes & 25/10/2025 & 25/10/2025 & Lucas & Diagrama de redes finalizado: app envia dados via HTTPS para API Node.js, que aciona microserviço Python (CNN) e armazena resultados em MySQL na nuvem. Inclui painel de gestão e observabilidade. \\ \hline

Pitch & 25/10/2025 & – & Lucas & Criação de novo pitch com vídeos dos integrantes em inglês e legendas em Português e Inglês. \\ \hline

\end{tabular}
\end{adjustbox}

\vspace{0.3em}
\small{\textbf{Fonte:} Equipe 21 – Vitalliz (2025)}

\end{center}