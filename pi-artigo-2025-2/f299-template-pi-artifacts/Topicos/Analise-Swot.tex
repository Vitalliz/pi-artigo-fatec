A análise SWOT foi realizada com o objetivo de compreender os principais
pontos fortes, fracos, oportunidades e ameaças relacionadas ao projeto NitrusLeaf.

\begin{table}[h]
\centering
\caption{Quadro Scrum com tarefas do projeto}
\renewcommand{\arraystretch}{1.2}
\begin{tabularx}{\textwidth}{|>{\raggedright\arraybackslash}X|>{\raggedright\arraybackslash}X|}
\hline
\textbf{Pontos Fortes} & \textbf{Fraquezas} \\
\hline
\begin{itemize}[left=0pt]
  \item É alinhado com iniciativas sustentáveis.
  \item Melhora a eficiência e rapidez no diagnóstico de doenças.
  \item Reduz o risco de pés de mexerica ainda saudáveis.
  \item Reduz o desperdício de alimentos saudáveis.
  \item É alinhado com os objetivos da ODS.
\end{itemize}
&
\begin{itemize}[left=0pt]
  \item Poucos estudos na área envolvendo plantas cítricas.
  \item O treinamento do modelo de IA depende de um grande banco de dados.
  \item O diagnóstico depende da qualidade da imagem.
  \item Dificuldade de acesso à internet por parte dos produtores.
\end{itemize}
\\
\hline
\textbf{Oportunidades} & \textbf{Ameaças} \\
\hline
\begin{itemize}[left=0pt]
  \item Expandir para outras deficiências como: zinco, ferro etc.
  \item Parcerias com empresas, cooperativas e universidades para divulgar o produto.
  \item Crescente demanda por agricultura de precisão.
\end{itemize}
&
\begin{itemize}[left=0pt]
  \item Resistência por parte dos produtores tradicionais.
  \item Concorrência com outras soluções, como drones, por exemplo.
\end{itemize}
\\
\hline
\end{tabularx}
\medskip
\SourceOrNote{Equipe 21 - Vitalliz (2025)}
\end{table}
\FloatBarrier
\medskip
Esse tipo de avaliação permite identificar elementos que podem impactar
diretamente o sucesso do sistema, além de apontar caminhos para melhorias contínuas
e decisões estratégicas.
